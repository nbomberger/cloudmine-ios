This is the native i\-O\-S S\-D\-K for Cloud\-Mine. It uses A\-R\-C (Automatic Reference Counting) and is thus {\bfseries compatible only with X\-Code 4.\-1 or higher and i\-O\-S 4 or higher.}

It has the following dependencies\-:


\begin{DoxyItemize}
\item C\-F\-Network
\item System\-Configuration
\item Mobile\-Core\-Services
\item Core\-Graphics
\item U\-I\-Kit
\item libz
\end{DoxyItemize}

You must also set the {\ttfamily -\/all\-\_\-load} and {\ttfamily -\/\-Obj\-C} flags in the {\bfseries Other Linker Flags} section of your app's build settings.

Watch the introductory \href{http://cloudmine.me/developer_zone#ios/tutorials}{\tt screencast} to see how to set up a new i\-O\-S project in X\-Code using the Cloud\-Mine framework, including how to specify all the dependencies.

Please see the \href{http://cloudmine.me/developer_zone#ios/overview}{\tt documentation overview} on our website for more details.

If you wish to simply download the precompiled universal framework, you \href{https://github.com/cloudmine/cloudmine-ios/downloads}{\tt may do so}.

\subsection*{Building}

To modify and build this framework yourself, simply open {\ttfamily cm-\/ios.\-xcworkspace} in X\-Code. Do not open any of the project files in the {\ttfamily ios/} directory directly as things won't work correctly.

There are a few schemes to pick from. Use {\ttfamily libcloudmine} for development work and for running the unit tests. All the unit tests are written using \href{https://github.com/allending/Kiwi/wiki}{\tt Kiwi}, a nice B\-D\-D-\/style unit testing framework. When you are ready to build the final framework for use in your own apps, choose the {\ttfamily Cloud\-Mine Universal Framework} scheme, clean, and build. This will build a universal framework that can run both on the i\-O\-S simulator as well as an i\-O\-S device. You can find the resulting framework under {\ttfamily ios/build/\-Release-\/iphoneuniversal}.

\subsection*{Contributing}

Contributions to the S\-D\-K are always welcome. However, please be sure you have well-\/written tests that cover all your cases. Since this is a framework, it is sometimes hard to test what you've written using unit tests. If that is the case for your contribution, write a small sample i\-Phone or i\-Pad application (it doesn't even need a U\-I) that demonstrates the correct, intended functionality of your additions to the framework. Once all that is done, submit a pull request clearly explaining your additions and providing links to the external test cases if applicable. If you have any questions, please contact the maintainer directly at \href{mailto:marc@cloudmine.me}{\tt marc@cloudmine.\-me}.

Thanks in advance for all your hard work and awesome code! \-:) 